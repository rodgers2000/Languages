%%%%%%%%%%%%%%%%%%%%%%%%%%%%%%%%%%%%%%%%%
% Jacobs Landscape Poster
% LaTeX Template
% Version 1.0 (29/03/13)
%
% Created by:
% Computational Physics and Biophysics Group, Jacobs University
% https://teamwork.jacobs-university.de:8443/confluence/display/CoPandBiG/LaTeX+Poster
% 
% Further modified by:
% Nathaniel Johnston (nathaniel@njohnston.ca)
%
% This template has been downloaded from:
% http://www.LaTeXTemplates.com
%
% License:
% CC BY-NC-SA 3.0 (http://creativecommons.org/licenses/by-nc-sa/3.0/)
%
%%%%%%%%%%%%%%%%%%%%%%%%%%%%%%%%%%%%%%%%%

%----------------------------------------------------------------------------------------
%	PACKAGES AND OTHER DOCUMENT CONFIGURATIONS
%----------------------------------------------------------------------------------------

\documentclass[final]{beamer}

\frenchspacing % only one space between sentences

\usepackage[scale=1.35]{beamerposter} % Use the beamerposter package for laying out the poster

\usetheme{confposter} % Use the confposter theme supplied with this template

\setbeamercolor{block title}{fg=ngreen,bg=white} % Colors of the block titles
\setbeamercolor{block body}{fg=black,bg=white} % Colors of the body of blocks
\setbeamercolor{block alerted title}{fg=white,bg=dblue!70} % Colors of the highlighted block titles
\setbeamercolor{block alerted body}{fg=black,bg=dblue!10} % Colors of the body of highlighted blocks
% Many more colors are available for use in beamerthemeconfposter.sty
\setbeamertemplate{logo}{}

%-----------------------------------------------------------
% Define the column widths and overall poster size
% To set effective sepwid, onecolwid and twocolwid values, first choose how many columns you want and how much separation you want between columns
% In this template, the separation width chosen is 0.024 of the paper width and a 4-column layout
% onecolwid should therefore be (1-(# of columns+1)*sepwid)/# of columns e.g. (1-(4+1)*0.024)/4 = 0.22
% Set twocolwid to be (2*onecolwid)+sepwid = 0.464
% Set threecolwid to be (3*onecolwid)+2*sepwid = 0.708

\newlength{\sepwid}
\newlength{\onecolwid}
\newlength{\twocolwid}
\newlength{\threecolwid}
\setlength{\paperwidth}{48in} % A0 width: 46.8in
\setlength{\paperheight}{36in} % A0 height: 33.1in
\setlength{\sepwid}{0.024\paperwidth} % Separation width (white space) between columns
\setlength{\onecolwid}{0.22\paperwidth} % Width of one column
\setlength{\twocolwid}{0.464\paperwidth} % Width of two columns
\setlength{\threecolwid}{0.708\paperwidth} % Width of three columns
\setlength{\topmargin}{-0.5in} % Reduce the top margin size

%-----------------------------------------------------------

\usepackage{graphicx}  % Required for including images

\usepackage{booktabs} % Top and bottom rules for tables

\usepackage{tikz}

\usepackage{hyperref}
\hypersetup{
    colorlinks = true,
    allcolors = {blue}
}
%----------------------------------------------------------------------------------------
%	TITLE SECTION 
%----------------------------------------------------------------------------------------

\title{My Undergraduate Projects} % Poster title

\author{Michael Rodgers} % Author(s)

\institute{Statistics Department, Rice University} % Institution(s)


%----------------------------------------------------------------------------------------

\begin{document}

\addtobeamertemplate{headline}{} 
{\begin{tikzpicture}[remember picture, overlay]
     \node [anchor=north west, inner sep=2cm]  at (current page.north west)
     {\includegraphics[height=9cm]{rice-logo.jpg}};
  \end{tikzpicture}}

\addtobeamertemplate{block end}{}{\vspace*{2ex}} % White space under blocks
\addtobeamertemplate{block alerted end}{}{\vspace*{2ex}} % White space under highlighted (alert) blocks

\setlength{\belowcaptionskip}{2ex} % White space under figures
\setlength\belowdisplayshortskip{2ex} % White space under equations

\begin{frame}[t] % The whole poster is enclosed in one beamer frame

\begin{columns}[t] % The whole poster consists of three major columns, the second of which is split into two columns twice - the [t] option aligns each column's content to the top

\begin{column}{\sepwid}\end{column} % Empty spacer column

\begin{column}{\onecolwid} % The first column

%----------------------------------------------------------------------------------------
%	OBJECTIVES
%----------------------------------------------------------------------------------------

\begin{alertblock}{Introduction}

This document contains all the projects I worked on during my undergraduate career. 
For each project, I will elaborate on my contribution and reflect upon my 
experience. I have posted my project code onto my GitHub account. The link is in 
contact information.

\end{alertblock}

%----------------------------------------------------------------------------------------
%	INTRODUCTION
%----------------------------------------------------------------------------------------

\begin{block}{MATH 212}

My instructor for Math 212 required us to do a LaTeX 
project. I did my project on double integrals. This 
problem uses Riemann integration, Fubini's theorem, 
and implicit substitution. This class exposed me to multivariate 
calculus, which led to an understanding of matrix calculus. 
\href{https://www.dropbox.com/s/bl541qefgk8hnfn/math-212.pdf?dl=0}
{Link}

\end{block}

\begin{block}{SMGT 430}

In my sports analytics course, I was required to do two projects. For my first project, I used conditional probabilities to show the frequency of a certain pitch being thrown given a specific previous pitch sequence. I used two methods: conditioning on the exact previous pitches and conditioning on the previous fastball or off-speed pitches. \href{https://www.dropbox.com/s/1lt6luvqioqada3/smgt-430-pitch-sequences.pdf?dl=0}
{Link}

For my second project, I used time series models to predict mlb's 2018 teams' seasonal records. I compared different models and used two ensemble methods: naive weights and after weights. In the after weights model, I learned the weights by minimizing an objective function. 
\href{https://www.dropbox.com/s/0shbgh0q1q4bwfh/smgt-430-record-prediction.pdf?dl=0}
{Link}


\end{block}

%------------------------------------------------
%----------------------------------------------------------------------------------------

\end{column} % End of the first column

\begin{column}{\sepwid}\end{column} % Empty spacer column

\begin{column}{\twocolwid} % Begin a column which is two columns wide (column 2)

\begin{columns}[t,totalwidth=\twocolwid] % Split up the two columns wide column

\begin{column}{\onecolwid}\vspace{-.6in} % The first column within column 2 (column 2.1)

%----------------------------------------------------------------------------------------
%	MATERIALS
%----------------------------------------------------------------------------------------

\begin{block}{STAT 405}

This project analyzed the four-seam fastball. We used 
MLB's Statcast 2016 data. We found that 68 percent of 
swinging strikes have a positive spin change. A 
positive spin change means that the four-seam fastball 
had an increasing spin rate, rather than a decreasing 
spin rate, on its way to home plate. The graphs were 
done in program R and I did all graphs and theory.
\href{https://www.dropbox.com/s/mv9xf1h63s1n5ww/stat-405.pdf?dl=0}
{Link}


\end{block}

%----------------------------------------------------------------------------------------

\end{column} % End of column 2.1

\begin{column}{\onecolwid}\vspace{-.6in} % The second column within column 2 (column 2.2)

%----------------------------------------------------------------------------------------
%	METHODS
%----------------------------------------------------------------------------------------

\begin{block}{STAT 413}

This project gave me experience applying machine 
learning algorithms to data. We tested all methods 
taught in class and came up with our own min-max
pipeline. This pipeline allows for classes to be 
separated, which creates the ability to build methods for each group. I came up with the min-max pipeline by using a for-loop to view all features and noticed some features had a significant amount of separation. 
\href{https://www.dropbox.com/s/d0ahud1j1f8ex5p/stat-413.pdf?dl=0}
{Link}

\end{block}

%----------------------------------------------------------------------------------------

\end{column} % End of column 2.2

\end{columns} % End of the split of column 2 - any content after this will now take up 2 columns width

%----------------------------------------------------------------------------------------
%	IMPORTANT RESULT
%----------------------------------------------------------------------------------------

\begin{alertblock}{Future Aspirations}

I am pursuing The Master in Computational Science and Engineering (M.C.S.E.) at Rice 
University in the Fall, 2018. My course work will focus on advancing my mathematical 
statistical skills and improving my computational skills for machine learning and 
code factoring. 

\end{alertblock} 

%----------------------------------------------------------------------------------------

\begin{columns}[t,totalwidth=\twocolwid] % Split up the two columns wide column again

\begin{column}{\onecolwid} % The first column within column 2 (column 2.1)

%----------------------------------------------------------------------------------------
%	MATHEMATICAL SECTION
%----------------------------------------------------------------------------------------

\begin{block}{STAT 450}

In this project, I generalized the concept of polynomial regression for 
classification problems. I developed a time-matrix that takes features and creates 
new features based on current and past time observations. This concept generalizes to any time-dependent data and can improve a base learners accuracy rate. 
\href{https://www.dropbox.com/s/njikgukcyu9roh3/stat_450.pdf?dl=0}
{Link}

\end{block}

\begin{block}{STAT 482}

This project used discrete returns, based on 
percentiles, and fundamental data to predict future 
returns. I did all the coding and theory for this project. I used 
lists to store the data and structured-modular-programming to 
implement the back-test. The PowerPoint in the link shows the setup. 
The machine learning algorithms performed well and illustrated the 
power they hold. We found that Random Forests with 20 labels yielded 
the highest CAGR (20\%).
\href{https://www.dropbox.com/s/z7owozay8apll1y/stat-482.pdf?dl=0}
{Link}

\end{block}

%----------------------------------------------------------------------------------------

\end{column} % End of column 2.1

\begin{column}{\onecolwid} % The second column within column 2 (column 2.2)

%----------------------------------------------------------------------------------------
%	RESULTS
%----------------------------------------------------------------------------------------

\begin{block}{STAT 486}

This project compared two option strategies to benchmark returns. The 
options backtest is from 1996-2015, and the benchmark backtest is from 
1970-2016. Santi Tellez created a program to backtest these option 
strategies, and I created a Shiny App to illustrate our findings. The 
shiny app allows you to compare these investing strategies in real 
time, with selecting your ideal investing conditions. The Shiny App 
uses graphics and statistics to compare these strategies. The options 
backtest is performed on the SPX, which has European style options. 
These options are cash settled and have weekly and monthly contracts.
\href{https://www.dropbox.com/s/u5spikswrc800rw/stat-486.pdf?dl=0}
{Link}

\end{block}

%---------------------------------------------------------------------
%-------------------

\end{column} % End of column 2.2

\end{columns} % End of the split of column 2

\end{column} % End of the second column

\begin{column}{\sepwid}\end{column} % Empty spacer column

\begin{column}{\onecolwid} % The third column

%	ADDITIONAL INFORMATION
%----------------------------------------------------------------------------------------

\begin{block}{STAT 606}

This project analyzed linear trends and rates of 
change to find patterns in price series. We compared 
our Moving Difference model to the Efficient Market 
Hypothesis. These are two non-parametric models. An 
ARIMA or SARIMA model would perform better. The Moving Difference 
model comes from an equation that I derived. All results were done in 
SAS. 
\href{https://www.dropbox.com/s/qoq73022kcygtn2/stat-606.pdf?dl=0}
{Link}

\end{block}

%----------------------------------------------------------------------------------------
%	CONCLUSION
%----------------------------------------------------------------------------------------

\begin{alertblock}{Conclusion}

These projects gave me experience applying statistics. They allowed me 
to master R and tested my creativity. To improve on my work, I am going to focus on advancing my math skills and improving my code factoring.

\end{alertblock}

%----------------------------------------------------------------------------------------

%----------------------------------------------------------------------------------------
%	REFERENCES
%----------------------------------------------------------------------------------------

\begin{block}{References}

\nocite{*} % Insert publications even if they are not cited in the poster
\small{\bibliographystyle{unsrt}
\bibliography{sample}\vspace{0.75in}}

\end{block}

%----------------------------------------------------------------------------------------
%	ACKNOWLEDGEMENTS
%----------------------------------------------------------------------------------------

\setbeamercolor{block title}{fg=red,bg=white} % Change the block title color

%----------------------------------------------------------------------------------------
%	CONTACT INFORMATION
%----------------------------------------------------------------------------------------

\setbeamercolor{block alerted title}{fg=black,bg=norange} % Change the alert block title colors
\setbeamercolor{block alerted body}{fg=black,bg=white} % Change the alert block body colors

\begin{alertblock}{Contact Information}

\begin{itemize}
\item Web: \href{https://www.linkedin.com/in/mikerodg}
{www.linkedin.com/in/mikerodg}
\item GitHub: \href{https://github.com/mjr2000}{https://github.com/mjr2000}
\item Email: \href{mailto:mjr10@rice.edu}
{mjr10@rice.edu}
\item Phone: 713-540-5357
\item Resume: \href{https://www.dropbox.com/s/nncfupjevd9gki1/RESUME.pdf?dl=0}{Link}
\end{itemize}

\end{alertblock}
%----------------------------------------------------------------------------------------

\end{column} % End of the third column

\end{columns} % End of all the columns in the poster

\end{frame} % End of the enclosing frame

\end{document}
